\documentclass{article}
\usepackage[UTF8]{ctex}
\usepackage{amsmath} % 数学数式排版
\usepackage{amssymb} % 数学符号
\usepackage{amsfonts} % 数学字体
\usepackage{geometry} % 调整页边距
\usepackage{amsthm}
\usepackage{bm}
\newtheorem{lemma}{lemma}
\newtheorem{theorem}{定理}[section]
% 设置页面布局
\geometry{
	a4paper,
	left=20mm,
	right=20mm,
	top=25mm,
	bottom=25mm
}

% 定义新命令,用于快速输入常用的数学符号
\newcommand{\R}{\mathbb{R}} % 实数集
\newcommand{\N}{\mathbb{N}} % 自然数集
\newcommand{\C}{\mathbb{C}} % 复数集
\newcommand{\Q}{\mathbb{Q}} % 有理数集
\newcommand{\Z}{\mathbb{Z}} % 整数集

\title{数学笔记}
\author{Your Name}
\date{\today}

\begin{document}
	\maketitle
	
	\section{引言}
	\begin{lemma}
		Given $f(x)$ is convex and L-smooth then
		\begin{equation}
		f(y)-f(x) \leq \nabla f(y)^T(y-x) - \frac{1}{2L}\Vert\nabla f(y)-\nabla f(x)\Vert^2_2\label{convLsmooth}
		\end{equation}
	\end{lemma}
	
	\begin{proof}[proof]
		Because of the convexity of $f(x)$,
			\begin{equation}
					f(z)\geq f(y)+\nabla f(y)^T(z-y)\\ 
					\Longleftrightarrow
					f(y)-f(z)\leq \nabla f(y)^T(y-z)\label{convexity}
			\end{equation}
		Because $f$ is $L$ smooth:
		\begin{equation*} 
			f(z)\leq f(x) + \nabla f(x)^T(z-x)+\frac{L}{2}\Vert z-x \Vert_2^2 \Longleftrightarrow
		\end{equation*}
		\begin{equation}
			f(z)-f(x)\leq \nabla f(x)^T(z-x)+\frac{L}{2}\Vert z - x \Vert_2^2\label{lsmooth}
		\end{equation}
		Using (\ref{convexity}) and (\ref{lsmooth}) we have:\\
		\begin{equation}
			f(y)-f(x) = f(y)-f(z) + f(z)-f(x)\leq \nabla f(y)^T(y-z) + \nabla f(x)^T(z-x) + \frac{L}{2}
			\Vert z - x \Vert_2^2
		\end{equation}
		let $z=x-\frac{1}{L}(\nabla f(x) - \nabla f(y))$:
		\begin{equation}
			f(y)-f(x) \leq \nabla f(y)^T(y-x) - \frac{1}{2L} \Vert \nabla f(x) - \nabla f(y) \Vert_2^2
		\end{equation}
	\end{proof}
	\begin{lemma}
		Suppose $F_k$ is L-smooth with global minimum at $\bm{w}^\ast_k$,then for any $\bm{w}_k$ in the domain of $F_k$,we have that
		\begin{equation}
		l = \Vert \nabla F_k(\bm{w}_k) \Vert_2^2 \leq 2L(F_k(\bm{{w}}_k) - F_k(\bm{w}_k^\ast))
		\end{equation}
		\begin{proof}[proof]
			Let $y=\bm{w}_k^\ast$ and $x = \bm{w}_k$ in (\ref{convLsmooth}), and $\nabla f(\bm{w}^\ast_k) = 0$
		\end{proof}
	\end{lemma}
	\section{重要定理}
	这里列举一些重要的定理,并附上其证明。
		
	\subsection{勾股定理}
	勾股定理给出了直角三角形的边之间的关系。它可以用以下公式表示:
	\begin{equation}
		a^2 + b^2 = c^2
	\end{equation}
	其中,$a$ 和 $b$ 是直角三角形的两条直角边的长度,$c$ 是斜边的长度。
	
	\textbf{证明:}(略)
	
	\subsection{欧拉公式}
	欧拉公式是数学中一条重要的公式,连接了五个基本数学常数:$e$(自然对数的底)、$\pi$(圆周率)、$i$(虚数单位)、$1$(单位元)和无穷远处的 $0$。它可以表示为:
	\begin{equation}
		e^{i\pi} + 1 = 0
	\end{equation}
	
	\textbf{证明:}(略)
	
	\section{结论}
	在这篇笔记中,我们总结了一些重要的数学定理和公式。
	
\end{document}
